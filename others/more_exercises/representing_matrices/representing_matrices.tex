\documentclass[a4paper,10pt,twoside,openany]{article}

\usepackage[lang=hebrew]{maths}
\usepackage{hebrewdoc}
\usepackage{stylish}
\usepackage{lipsum}
\let\bs\blacksquare

\setlength{\parindent}{0pt}

%%%%%%%%%%%%
% Styling %
%%%%%%%%%%%%

\usepackage{enumitem}

%%%%%%%%%%%%%
% Counters  %
%%%%%%%%%%%%%

\setcounter{section}{1}

%%%%%%%%%%
% Title  %
%%%%%%%%%%
\title{
אלגברה ב' (104168)  \\
תרגילים על מטריצות מייצגות
}

\author{אלן סורני}

\begin{document}
\maketitle

\begin{notation}
עבור מרחבים וקטוריים
$V,W$
מעל אותו שדה
$\mbb{F}$
נסמן ב־%
$\Hom_{\mbb{F}}\prs{V,W}$
את אוסף ההעתקות הלינאריות
$V \to W$,
ונסמן
$\End_{\mbb{F}}\prs{V} \coloneqq \Hom_{\mbb{F}}\prs{V,V}$.
לעתים נשמיט את כתיבת השדה
$\mbb{F}$
מהסימון, כשהוא ידוע או לא חשוב בקונטקסט.
\end{notation}

\begin{notation}
יהיו
$V,W$
מרחבים וקטוריים סוף־מימדיים מעל אותו שדה
$\mbb{F}$
עם בסיסים
$B,C$
בהתאמה
ותהי
$T \in \Hom_{\mbb{F}}\prs{V,W}$.
אם
$B = \prs{v_1, \ldots, v_n}$,
נגדיר
\[\text{.} \brs{T}^B_C = \pmat{\vert & & \vert \\ \brs{T\prs{v_1}}_C & \cdots & \brs{T\prs{v_n}}_C \\ \vert & & \vert}\]
\end{notation}

\begin{exercise}
עבור ההעתקות והבסיסים הבאים, כיתבו את המטיצה המייצגת
$\brs{T}^B_C$.

\begin{enumerate}
\item
\begin{align*}
T \colon \mbb{R}^3 &\mapsto \mbb{R}^3 \\
\pmat{x\\y\\z} &\mapsto \pmat{2x+z\\5x-3y\\2y+3z}
\end{align*}
והבסיסים
\begin{align*}
B &= \prs{e_1, e_2, e_3} \coloneqq \prs{\pmat{1 \\ 0 \\ 0}, \pmat{0 \\ 1 \\ 0}, \pmat{0 \\ 0 \\ 1}} \\
C &= \prs{e_1+e_2, e_2 + e_3, e_3}
\end{align*}

\item \begin{align*}
T \colon \mbb{R}_3\brs{x} &\to \mbb{R}_3\brs{x} \\
p\prs{x} &\mapsto p'\prs{x}
\end{align*}
והבסיסים
\begin{align*}
B &= \prs{1,x,x^2,x^3} \\
C &= \prs{1+x,x+x^2,x^2+x^3,x^3}
\end{align*}
כאשר
$\mbb{R}_3\brs{x}$
מרחב הפולינום הממשיים ממעלה לכל היותר
$3$.

\item
\begin{align*}
T \colon \Mat_{2,2}\prs{\mbb{C}} &\to \Mat_{2,2}\prs{\mbb{C}} \\
A &\mapsto A^t
\end{align*}
והבסיסים
\begin{align*}
B = \prs{E_{1,1}, E_{1,2}, E_{2,1}, E_{2,2}} &\coloneqq \prs{\pmat{1 & 0 \\ 0 & 0}, \pmat{0 & 1 \\ 0 & 0}, \pmat{0&0\\1&0}, \pmat{0&0\\0&1}} \\
\text{.} C &= \prs{\pmat{0&1\\1&1},\pmat{1&0\\1&1},\pmat{1&1\\0&1},\pmat{1&1\\1&0}}
\end{align*}
\end{enumerate}
\end{exercise}

\begin{exercise}
תהי
\begin{align*}
\text{.} A &= \pmat{0 & 0 & 1 & 0 \\ 0 & 0 & 0 & 1 \\ 0 & 1 & 0 & 0 \\ 1 & 0 & 0 & 0} \in \Mat_{4,4}\prs{\mbb{C}}
\end{align*}
מיצאו בכל אחד מהמקרים הבאים בסיס
$C$
של
$V$
עבורו יתקיים
$\brs{\id_V}^B_C = A$.

\begin{enumerate}
\item $V = \mbb{R}_3\brs{x}$
וגם
$B = \pmat{1,x,x^2,x^3}$.

\item $V = \Mat_{2,2}\prs{\mbb{C}}$
וגם
$B = \prs{E_{1,1}, E_{2,2}, E_{1,2}, E_{2,1}}$.

\item $V = \mbb{C}^4$
וגם
$B = \prs{e_1 + 3e_3, e_2+3e_4, e_3+3e_1, e_4+3e_2}$.
\end{enumerate}
\end{exercise}

\begin{exercise}
תהי
\[\text{.} A = \pmat{1 & 0 & 1 & 0 \\ 0 & 2 & 0 & 1 \\ 0 & 1 & 0 & 4 \\ 1 & 0 & 0 & 0}\]
מיצאו בכל אחד מהמקרים הבאים בסיס
$B$
של
$V$
עבורו יתקיים
$\brs{\id_V}^B_C = A$.

\begin{enumerate}
\item $V = \mbb{R}_3\brs{x}$
וגם
$B = \pmat{1,x,x^2,x^3}$.

\item $V = \Mat_{2,2}\prs{\mbb{C}}$
וגם
$B = \prs{E_{1,1}, E_{2,2}, E_{1,2}, E_{2,1}}$.

\item $V = \mbb{C}^4$
וגם
$B = \prs{e_1 + 3e_3, e_2+3e_4, e_3+3e_1, e_4+3e_2}$.
\end{enumerate}
\end{exercise}

\begin{exercise}
בכל אחד מהמקרים הבאים, מיצאו בסיס
$B$
של
$V$
עבורו
$\brs{T}^B_C = I$
מטריצת היחידה.

\begin{enumerate}
\item $V = \mbb{C}^4$
עם
$B = \prs{e_1, e_2, e_3, e_4}$
וגם
\begin{align*}
T \colon V &\to V \\
\text{.} \pmat{w\\x\\y\\z} &\mapsto \pmat{x\\y\\z\\w}
\end{align*}

\item $V = \mbb{C}_3\brs{x}$
עם
$B = \prs{1,x^2,x,x^3}$
וגם
\begin{align*}
T \colon V &\to V \\
\text{.} \prs{T\prs{p}}\prs{x} &= p\prs{x+1}
\end{align*}

\item $V = \Mat_{2,2}\prs{\mbb{C}}$
עם
$B = \prs{E_{1,1}, E_{2,2}, E_{1,2}, E_{2,1}}$
וגם
\begin{align*}
T \colon V &\to V \\
\text{.} A &\mapsto -A^t
\end{align*}
\end{enumerate}
\end{exercise}

\pagebreak

\begin{exercise}
בכל אחד מהמקרים הבאים, מיצאו בסיס
$C$
של
$V$
עבורו
$\brs{T}^B_C = I$
מטריצת היחידה.

\begin{enumerate}
\item $V = \mbb{C}^4$
עם
$C = \prs{e_1, e_2, e_3, e_4}$
וגם
\begin{align*}
T \colon V &\to V \\
\text{.} \pmat{w\\x\\y\\z} &\mapsto \pmat{x\\y\\z\\w}
\end{align*}

\item $V = \mbb{C}_3\brs{x}$
עם
$C = \prs{1,x^2,x,x^3}$
וגם
\begin{align*}
T \colon V &\to V \\
\text{.} \prs{T\prs{p}}\prs{x} &= p\prs{x+1}
\end{align*}

\item $V = \Mat_{2,2}\prs{\mbb{C}}$
עם
$C = \prs{E_{1,1}, E_{2,2}, E_{1,2}, E_{2,1}}$
וגם
\begin{align*}
T \colon V &\to V \\
\text{.} A &\mapsto -A^t
\end{align*}
\end{enumerate}
\end{exercise}

\begin{exercise}
בכל אחד מהמקרים הבאים, מצאו אופרטור
$T \in \End_{\mbb{F}}\prs{V}$
עבורו
$\brs{T}_B = A$.

\begin{enumerate}
\item $V = \mbb{C}^4$
עם
$B = \prs{e_1+e_2,e_2+e_4,e_3,e_3+e_4}$
וגם
$A = \pmat{0 & 1 & 0 & 0 \\ 0 & 0 & 1 & 0 \\ 0 & 0 & 0 & 1 \\ 0 & 0 & 0 & 0}$.

\item $V = \mbb{C}_2\brs{x}$
עם
$B = \prs{1,x^2,x+x^2}$
וגם
$A = \pmat{2 & 0 & 2 \\ 0 & 2 & 0 \\ 2 & 0 & 2}$.

\item $V = \Mat_{2,2}\prs{\mbb{R}}$
עם
$B = \prs{\pmat{0&1\\1&1},\pmat{1&0\\1&1},\pmat{1&1\\0&1},\pmat{1&1\\1&0}}$
וגם
$A = \pmat{1 & 1 & 1 & 1 \\ 1 & 1 & 1 & 1 \\ 1 & 1 & 1 & \\ 1 & 1 & 1 & 1}$.
\end{enumerate}
\end{exercise}

\end{document}