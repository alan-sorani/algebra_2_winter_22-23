\documentclass[a4paper,10pt,twoside,openany]{article}

\usepackage[lang=hebrew]{maths}
\usepackage{hebrewdoc}
\usepackage{stylish}
\usepackage{lipsum}
\let\bs\blacksquare

\setlength{\parindent}{0pt}

%%%%%%%%%%%%
% Styling %
%%%%%%%%%%%%

\usepackage{enumitem}

%%%%%%%%%%%%%
% Counters  %
%%%%%%%%%%%%%

\setcounter{section}{1}

%%%%%%%%%%
% Title  %
%%%%%%%%%%
\title{
אלגברה ב' (104168)  \\
תרגילים ממבחנים על דטרמיננטות ועל צורת ובסיס ז'ורדן
}

\author{אלן סורני}

\begin{document}
\maketitle

\begin{exercise}[מועד א' - אביב 2020]
תהי
$A \in \Mat_n\prs{\mbb{F}}$
נילפוטנטית. הראו כי
$\det\prs{A} = 0$
וגם
$\det\prs{I+A} = 1$.
\end{exercise}

\begin{exercise}[מועד ב' - אביב 2020]
תהיינה
$A,B \in \Mat_{3}\prs{\mbb{R}}$
עבורן מתקיים
\[\text{.} \det\prs{A} = \det\prs{A+B} = \det\prs{A + 2B} = \det\prs{A + 3B} = 1\]
מיצאו את
$\det\prs{3A + 7B}$.
\end{exercise}

\begin{exercise}[מועד א' - חורף 2022]
תהי
\[\text{.} A \coloneqq \pmat{0 & -1 & -1 & 0 \\ 1 & -2 & -1 & -1 \\ 0 & 0 & -1 & 1 \\ 0 & 0 & 0 & -1} \in \Mat_4\prs{\mbb{C}}\]
\begin{enumerate}
\item מיצאו את צורת ז'ורדן של
$A$
ומיצאו מטריצה הפיכה
$P \in \Mat_4\prs{\mbb{C}}$
עבורה
$P^{-1} A P$
מטריצת ז'ורדן.

\item הוכיחו כי
$A^{-1}$
ו־%
$A^3$
דומות.
\end{enumerate}
\end{exercise}

\begin{exercise}[מועד א' - אביב 2021]
\begin{enumerate}
\item
יהי
$\lambda \in \mbb{C} \setminus \set{0}$
ויהי
$n > 1$.
הראו כי
$J_n\prs{\lambda}^2$
אינה לכסינה.

\item
יהי
$V$
מרחב וקטורי מרוכב סוף־מימדי, יהי
$T \in \End_{\mbb{C}}\prs{V}$
כך ש־%
$T^2$
לכסין. מיצאו את כל צורות ז'ורדן האפשריות של
$T$.
\end{enumerate}
\end{exercise}

\begin{exercise}[מועד א' - אביב 2022]
יהי
$V$
מרחב וקטורי סוף־מימדי מעל
$\mbb{C}$,
ויהי
$T \in \End_{\mbb{C}}\prs{V}$.
\begin{enumerate}
\item בהינתן
$p \in \mbb{C}\brs{x}$,
הראו שקיים
$S \in \End_{\mbb{C}}\prs{V}$
עבורו
$p\prs{S} = T$,
אם ורק אם קיים בסיס
$B$
של
$V$
וקיימת מטריצת ז'ורדן
$J$
עבורה
\[\text{.} \brs{T}_B = p\prs{J}\]

\item
נניח כי
$\dim V = 4$
וכי הפולינום המינימלי של
$T$
הוא
$m_T\prs{x} = x^3$.
הוכיחו כי אין
$S \in \End_{\mbb{C}}\prs{V}$
עבורו
$S^2 = T$.
\item
תנו דוגמה מפורשת למפורטת לאופרטור
$S \in \End_{\mbb{C}}\prs{\mbb{C}^4}$
עבורו
$\dim \ker \prs{S^2} = 2$
וגם
$\dim \ker \prs{S^4} = 3$.
\end{enumerate}
\end{exercise}

\begin{exercise}[מועד א' - חורף 2021]
הראו כי קיימת מטריצה
$A \in \Mat_{3}\prs{\mbb{R}}$
עבורה
\[\text{.} A^{20} + A^{21} = \pmat{1 & 20 & 0 \\ 0 & 1 & 21 \\ 0 & 0 & 1}\]

\textbf{רמז:}
ראינו בהרצאה כי
\[\text{.} J_n\prs{\lambda}^k = \sum_{i = 0}^k \binom{k}{i} \lambda^{k-i} J_n\prs{0}^i\]
\end{exercise}

\end{document}