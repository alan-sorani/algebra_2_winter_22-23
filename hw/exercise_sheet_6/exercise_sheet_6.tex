\documentclass[a4paper,10pt,twoside,openany]{article}

\usepackage[lang=hebrew]{maths}
\usepackage{hebrewdoc}
\usepackage{stylish}
\usepackage{lipsum}
\let\bs\blacksquare

\setlength{\parindent}{0pt}

%%%%%%%%%%%%
% Styling %
%%%%%%%%%%%%

\usepackage{enumitem}

%%%%%%%%%%%%%
% Counters  %
%%%%%%%%%%%%%

\setcounter{section}{1}     
            
%BIBLIOGRAPHY
\usepackage[
backend=biber,
style=alphabetic,
]{biblatex}
\addbibresource{bibliography.bib} %Imports bibliography file

%%%%%%%%%%
% Title  %
%%%%%%%%%%
\title{
אלגברה ב' - גיליון תרגילי בית 6 \\
חוק האינרציה של סילבסטר, וקריטריון סילבסטר
\\
\vspace{1cm}
\large{תאריך הגשה: 26.1.2023}
}
\date{}

\begin{document}
\maketitle

\begin{exercise}
תהי
$A \in \Mat_n\prs{\mbb{R}}$
סימטרית ויהי
$m$
סכום הריבויים האלגבריים של הערכים העצמיים החיוביים של
$A$.

\begin{enumerate}
\item הראו כי יש תת־מרחב
$W \leq \mbb{R}^n$
ממימד
$m$
עבורו
$\trs{Aw, w} > 0$
לכל
$w \in W \setminus \set{0}$.

\item יהי
$W' \leq \mbb{R}^n$
תת־מרחב נוסף עבורו מתקיים
$\trs{Aw, w} > 0$
לכל
$w \in W' \setminus \set{0}$.
הראו כי
$\dim W' \leq m$.
\end{enumerate}
\end{exercise}

\begin{exercise}
יהי
$V = \Mat_2\prs{\mbb{R}}$.

\begin{enumerate}
    \item מיצאו מכפלה פנימית
    $f$
    על
    $V$
    כך ש־%
    \[B = \pmat{E_{1,1} + E_{1,2}, E_{1,2} + E_{2,1}, E_{2,1} + E_{2,2}, E_{2,2}}\]
    הוא בסיס אורתונורמלי לפי
    $f$.
    
    \item
    תהי
    \[g\prs{A,B} = \tr\prs{AB}\]
    תבנית בילינארית סימטרית על
    $V$.
    מיצאו בסיס
    $C$
    של
    $V$
    ומטריצה
    $S = \diag\prs{I_{n_+}, I_{n_-}, 0_{n_0}}$
    עבורם
    $\brs{g}_C = S$.
    
    \item
    מיצאו בסיס
    $D$
    של
    $V$
    עבורו
    $\brs{f}_D, \brs{g}_D$
    מטריצות אלכסוניות.
\end{enumerate}
\end{exercise}

\begin{exercise}
קיבעו אילו מהמטריצות הממשיות הבאות מוגדרות חיובית לחלוטין ואילו מוגדרות שלילית לחלוטין.

\begin{align*}
    A &= \pmat{1 & 2 & 0 \\ 2 & 3 & 1 \\ 0 & 1 & 3} \\
    B &= \pmat{1 & 0 & 1 & 1 \\ 0 & 1 & 1 & 0 \\ 1 & 1 & 3 & 1 \\ 1 & 0 & 1 & 3} \\
    C &= \pmat{-1 & 0 & 1 & 1 \\ 0 & -1 & 1 & 1 \\ 1 & 1 & -3 & -1 \\ 1 & 1 & -1 & -4} \\
    D &= \pmat{1 & 0 & 1 & 1 \\ 0 & 1 & 1 & 0 \\ 1 & 1 & 0 & 1 \\ 1 & 0 & 1 & -1}
\end{align*}
\end{exercise}

\begin{exercise}[רשות]
יהי
$V = \Hom_{\mbb{R}}\prs{\mbb{R}^4, \mbb{R}}$
ועבור כל
$a \in \mbb{R}$
תהי
\begin{align*}
    g_a\prs{\phi} &= \phi\prs{e_1}^2 + a\phi\prs{e_1}\phi\prs{e_2} + 2\phi\prs{e_1}\phi\prs{e_3} + 2\phi\prs{e_2}^2 + 2 \phi\prs{e_2}\phi\prs{e_3} + \phi\prs{e_3}^2 + 2 \phi\prs{e_3} \phi\prs{e_4} 
\end{align*}
תבנית ריבועית על
$V$.

\begin{enumerate}
    \item 
    לכל
    $a \in \mbb{R}$,
    מיצאו תבנית ריבועית
    $f_a$
    על
    $V$
    עבורה
    $g_a\prs{\phi} = f_a\prs{\phi, \phi}$.
    
    \item מיצאו את כל ערכי
    $a \in \mbb{R}$
    עבורם
    $g_a$
    מוגדרת חיובית לחלוטין.
    
    \item
    נניח כעת כי
    $a = 2$.
    מיצאו בסיס
    $B$
    של
    $V$
    עבורו
    $\brs{g}_B$
    מטריצה מהצורה
    $\pmat{I_{n_+} & 0 & 0 \\ 0 & - I_{n_-} & 0 \\ 0 & 0 & 0_{n_0}}$.
\end{enumerate}
\end{exercise}

\begin{exercise}[רשות]
תהי
\[A = \pmat{-1&1&1\\1&-1&1\\1&1&-1} \in M_3\prs{\mbb{R}}\]
ותהי
\[g_A\prs{v} = v^t A v\]
תבנית ריבועית על
$\mbb{R}^3$.

\begin{enumerate}
    \item מיצאו את הסימן של התבנית
    $g_A$.
    
    \item מיצאו וקטורים
    $v_1, v_2 \in \mbb{R}^3$
    מנורמה
    $1$
    עבורם
    $g_A\prs{v_1}$
    מינימלי ו־%
    $g_A\prs{v_2}$
    מקסימלי.
\end{enumerate}
\end{exercise}

\end{document}