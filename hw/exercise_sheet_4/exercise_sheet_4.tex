\documentclass[a4paper,10pt,twoside,openany]{article}

\usepackage[lang=hebrew]{maths}
\usepackage{hebrewdoc}
\usepackage{stylish}
\usepackage{lipsum}
\let\bs\blacksquare

\setlength{\parindent}{0pt}

%%%%%%%%%%%%
% Styling %
%%%%%%%%%%%%

\usepackage{enumitem}

%%%%%%%%%%%%%
% Counters  %
%%%%%%%%%%%%%

\setcounter{section}{1}     
            
%BIBLIOGRAPHY
\usepackage[
backend=biber,
style=alphabetic,
]{biblatex}
\addbibresource{bibliography.bib} %Imports bibliography file

%%%%%%%%%%
% Title  %
%%%%%%%%%%
\title{
אלגברה ב' - גיליון תרגילי בית 4 \\
מרחבי מכפלה פנימית
\\
\vspace{1cm}
\large{תאריך הגשה: 28.12.2022}
}
\date{}

\begin{document}
\maketitle

\begin{exercise}%1
יהי
$V = \mbb{R}_2\brs{x}$
ותהיינה
\begin{align*}
\trs{f,g}_1 &= \int_0^1 f\prs{x} g\prs{x} \diff x \\
\trs{f,g}_2 &= f\prs{-1} g\prs{-1} + f\prs{0} g\prs{0} + f\prs{1} g\prs{1}
\end{align*}
שתי מכפלות פנימיות על
$V$.
יהי
\[\text{.} W = \set{f \in V}{f\prs{x} = f\prs{-x}} \leq V\]
\begin{enumerate}
\item 
מיצאו בסיס
$B = \prs{w_1, \ldots, w_m}$
של
$W$
והשלימו אותו לבסיס
$C$
של
$V$.
בצעו את תהליך גרם־שמידט על
$C$
לפי כל אחת מהמכפלות הפנימיות כדי לקבל בסיסים אורתונורמליים לפיהן.

\item
היעזרו בבסיסים שמצאתן בסעיף הקודם כדי למצוא
את
$W^\perp$
לפי כל אחת מהמכפלות הפנימיות.

\item
מיצאו את ההטלה האורתוגונלית
$P_W$
על
$W$
לפי כל אחת מהמכפלות הפנימיות.

\item
יהי
$f\prs{x} = 1 + x$.
מיצאו את המרחק של
$f$
מ־%
$W$
לפי כל אחת מהמכפלות הפנימיות.
\end{enumerate}
\end{exercise}

\begin{exercise}%2
יהי
$V$
מרחב מכפלה פנימית סוף־מימדי, ויהי
$P \in \End_{\mbb{F}}\prs{V}$
המקיים
$P^2 = P$.

נניח כי
$\norm{P\prs{v}} \leq \norm{v}$
לכל
$v \in V$.
הראו כי
$\im\prs{P} \perp \ker\prs{P}$
והסיקו כי
$P$
הטלה אורתוגונלית.

\textbf{רמז:}
ראינו תרגיל שקישר בין אי־שוויון בין נורמות לבין ניצבות.
\end{exercise}

\begin{exercise}%3
יהי
$V$
מרחב מכפלה פנימית סוף־מימדי מעל
$\mbb{C}$,
ויהי
$T \in \End_{\mbb{C}}\prs{V}$.
הראו כי קיים בסיס אורתונורמלי
$B$
עבורו
$\brs{T}_B$
משולשת עליונה.

\textbf{רמז:}
היעזרו במשפטי ז'ורדן וגרם־שמידט.
\end{exercise}

\begin{exercise}%4
יהי
$V = M_2\prs{\mbb{R}}$
עם הבסיס
\[\text{.} B = \prs{\pmat{1 & 1 \\ 0 & 0}, \pmat{1 & 2 \\ 0 & 0}, \pmat{1 & 0 \\ 1 & 0}, \pmat{0 & 1 \\ 0 & 1}}\]
מיצאו מכפלה פנימית על
$V$
לפיה
$B$
בסיס אורתונורמלי.

\textbf{רמז:}
עבור בסיס
$C$
של
$V$
הגדרנו מכפלה פנימית על ידי
\[\text{,} \trs{u,v}_C = \trs{\brs{u}_C, \brs{v}_C}_{\mrm{Std}}\]
וראינו שכל המכפלות הפנימיות על
$V$
הן מהצורה הזאת.
\end{exercise}

\begin{exercise}%5
יהי
$V = \mbb{C}_4\brs{x}$
עם המכפלה הפנימית
\[\text{.} \trs{f,g} = \sum_{i \in \brs{4}} f\prs{i} \bar{g}\prs{i}\]
מיצאו
$g \in V$
עבורו
$\trs{f,g} = f\prs{-1}$
לכל
$f \in V$.
\end{exercise}

\end{document}